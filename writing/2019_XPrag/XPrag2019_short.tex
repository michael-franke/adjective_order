\documentclass[12pt]{article}
\usepackage[hmargin={1in},vmargin={1in}]{geometry}   
% \geometry{a4}              
\usepackage{color,graphicx}
\usepackage{setspace}
\usepackage{amsmath}
\usepackage{amssymb}
\usepackage{nicefrac}
% \usepackage{varioref}
\usepackage{textcomp}
\usepackage{textcomp}
\usepackage{mflogo}
\usepackage{wasysym}
\usepackage[normalem]{ulem}
\usepackage{hyperref}
\usepackage{booktabs}
\usepackage{linguex}
\usepackage{qtree}
\qtreecenterfalse

\newcommand{\HRule}{\rule{\linewidth}{0.25mm}}

\usepackage{fancyhdr} % This should be set AFTER setting up the page geometry
\pagestyle{plain} % options: empty , plain , fancy
\lhead{}\chead{}\rhead{}
\renewcommand{\headrulewidth}{.5pt}
\lfoot{}\cfoot{\thepage}\rfoot{}
\newcommand{\txtp}{\textipa}
\renewcommand{\rm}{\textrm}
\newcommand{\sem}[1]{\mbox{$[\![$#1$]\!]$}}
\newcommand{\den}[1]{\ensuremath{[\![#1]\!]}}
\newcommand{\lam}{$\lambda$}
\newcommand{\lan}{$\langle$}
\newcommand{\ran}{$\rangle$}
\newcommand{\type}[1]{\ensuremath{\left \langle #1 \right \rangle }}

\newcommand{\bex}{\begin{exe}}
\newcommand{\eex}{\end{exe}}
\newcommand{\bit}{\begin{itemize}}
\newcommand{\eit}{\end{itemize}}
\newcommand{\ben}{\begin{enumerate}}
\newcommand{\een}{\end{enumerate}}

\newcommand{\gcs}[1]{\textcolor{blue}{[gcs: #1]}}

%%%%%%%%%%%%%%%%%%%%%%%%%%%%%%%%%%%%%%%%%%%%%%%%%%%%%%
%% for a compact representation of the references 
%%%%%%%%%%%%%%%%%%%%%%%%%%%%%%%%%%%%%%%%%%%%%%%%%%%%%%

\usepackage[backend=biber,style=numeric,maxnames=2,natbib=true]{biblatex}
\addbibresource{adjOrder.bib}


\DeclareBibliographyDriver{std}{%
  \usebibmacro{bibindex}%
  \usebibmacro{begentry}%
  \usebibmacro{author/editor+others/translator+others}%
  \setunit{\labelnamepunct}\newblock
  \usebibmacro{title}%
  \newunit\newblock
  \usebibmacro{date}%
  \newunit\newblock
  \usebibmacro{finentry}}


\DeclareBibliographyAlias{article}{std}
\DeclareBibliographyAlias{book}{std}
\DeclareBibliographyAlias{booklet}{std}
\DeclareBibliographyAlias{collection}{std}
\DeclareBibliographyAlias{inbook}{std}
\DeclareBibliographyAlias{incollection}{std}
\DeclareBibliographyAlias{inproceedings}{std}
\DeclareBibliographyAlias{manual}{std}
\DeclareBibliographyAlias{misc}{std}
\DeclareBibliographyAlias{online}{std}
\DeclareBibliographyAlias{patent}{std}
\DeclareBibliographyAlias{periodical}{std}
\DeclareBibliographyAlias{proceedings}{std}
\DeclareBibliographyAlias{report}{std}
\DeclareBibliographyAlias{thesis}{std}
\DeclareBibliographyAlias{unpublished}{std}
\DeclareBibliographyAlias{*}{std}

\renewcommand*{\bibfont}{\footnotesize}
\defbibheading{bibliography}[\refname]{}

%%%%%%%%%%%%%%%%%%%%%%%%%%%%%%%%%%%%%%%%%%%%%%%%%%%%%%
%%%%%%%%%%%%%%%%%%%%%%%%%%%%%%%%%%%%%%%%%%%%%%%%%%%%%%


\thispagestyle{plain}

\begin{document}

\setlength{\abovedisplayskip}{0.5pt}
\setlength{\belowdisplayskip}{0.5pt}

%\maketitle

\begin{center}
	\textbf{The evolutionary pressure for subjectivity-based adjective ordering preferences}
  \vspace*{-0.4cm}
  \begin{flushright}
  {\footnotesize Michael Franke (Osnabr\"uck), Gregory Scontras (Irvine), Mihael Simoni\v{c} (Ljubljana)}
\end{flushright}
\end{center}

\vspace{-5pt}

\noindent 
There are strong, cross-linguistically attested preferences on the ordering of adjectives when
they modify a noun (e.g., \emph{big brown bag} vs.~ \emph{brown big bag}).
\cite{scontrasetal2017adjectives} provide empirical evidence that adjective order in
multi-adjective strings is reliably predicted by the subjectivity of the adjectives involved:
less subjective adjectives are preferred closer to the modified noun. However, the reasons for that
finding were not clear. This paper argues for an evolutionary rationale for these ordering
preferences. Building on \cite{scontrasetalSPadjectives} and \cite{simonic2018}, we argue that
there is a communicative advantage to ordering adjectives by their subjectivity. The main
intuition is that more subjective adjectives are prone to incur more inter-interlocutor
disagreement about whether the adjective applies to a given object or not. Therefore, to
minimize coordination error in referential communication, a particular way of ordering
adjectives is less error prone. We use mathematical modeling to flesh out this basic intuition
in a novel way, and resort to numerical simulations to demonstrate the intuited difference in
expected communicative success when averaging over a large number of randomly-generated
contexts.

\noindent \textbf{Semantics.} We adopt a context-dependent threshold-based semantics for
gradable adjectives, which \cite{schmidtetal2009} showed was the simplest highly-successful predictor of human judgements. The set $\den{\text{tall}}^C$ of objects in context
$C$ that count as tall in $C$ are those whose degree of tallness exceeds the threshold $\theta
= k(\text{tallest}(C) - \text{shortest}(C))$ (where $k$ is a free parameter in our modeling and
$\text{tallest}(C)$ / $\text{shortest}(C)$ are the heights of the tallest and shortest elements
in $C$). Following \cite{scontrasetalSPadjectives} and \cite{simonic2018}, we assume that
iterated adjectival modification triggers \emph{sequentially intersective context updates}.
Adjectives syntactically farther from the noun are interpreted relative to contexts that are
already restricted by closer adjectives. For example, the denotation of ``[adj\textsubscript{i}
[adj\textsubscript{j} $N$]]'' given a shared context $C$ of potential referents is
$\den{\text{[adj\textsubscript{i} [adj\textsubscript{j} $N$]]}}^{C} =
\den{\text{adj}_i}^{\den{\text{adj}_j}^{C\cap \den{N}}}$. The effect is that adjectives closer
to the noun will operate over a larger context (i.e., one that is less restricted).

\noindent \textbf{Simulations.} We use Monte Carlo simulation to estimate the expected
referential success, $\text{ES}(o) = \int P(C) \ \text{RefSuc}(ord,C) \ \text{d}C$, of
different orderings $o$ of adjective sequences when applied across many contexts $C$. We fix
two adjectival properties; one is more subjective than the other. We randomly generate contexts $C$
repeatedly (varying context size, difference in subjectivity etc.). Interlocutors have
noise-perturbed representations of $C$. The more subjective a property is, the bigger the
difference to the actual $C$. Interlocutors base their interpretation of phrases on their
individual subjective representations of the context. Based on this setup, we show that
$\text{ES}(o)$ is reliably and robustly higher for a sequence like ``big brown bag'' than
for``brown big bag''. This suggests that a simple, empirically-motivated semantics can lead to
increased communicative success when multi-adjective strings are ordered with respect to
decreasing subjectivity. We thus have an answer for the question of why subjectivity should
matter in adjective ordering: ordering adjectives by decreasing subjectivity increases
communicative success.

\printbibliography

\end{document}















