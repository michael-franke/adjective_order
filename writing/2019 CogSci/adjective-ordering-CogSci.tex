\documentclass[10pt,a4paper]{article}
\usepackage{tikz} % for drawing figures
\usepackage{amsmath} % for equations
\usepackage{url} % for URLs
\usepackage{graphicx}
\usepackage{multicol}
\usepackage{varwidth}
\usepackage{blindtext}


\usepackage{linguex} % ** special include in directory: for doing handy example labeling and bracketing
\renewcommand{\firstrefdash}{} % used for linguex package not to put hyphens in example refs (1a instead of 1-a)
\usepackage{cogsci}
\usepackage{pslatex}
\usepackage{apacite}

\newcommand{\sem}[1]{\mbox{$[\![$#1$]\!]$}}
\newcommand{\lam}{$\lambda$}
\newcommand{\gcs}[1]{\textcolor{blue}{[gcs: #1]}} 



\title{our title}
\author{\large \textbf{our names}\\
our emails\\
our affiliations}


\begin{document}
\maketitle

\begin{abstract}
Adjective ordering preferences (e.g., \emph{big blue box} vs.~\emph{blue big box}) are robustly attested in English and many unrelated languages \cite{dixon1982}. \citeA{scontrasetal2017adjectives} showed that adjective subjectivity is a robust predictor of ordering preferences in English: less subjective adjectives are preferred closer to the modified noun. In a follow-up to this empirical finding, \citeA{simonic2018} and \citeauthor{scontrasetalSPadjectives} (to appear) claim that pressures from successful reference resolution and the hierarchical structure of modification explain subjectivity-based ordering preferences. We provide further support for this claim using large-scale simulations of reference scenarios, together with an empirically-motivated adjective semantics. In the vast majority of cases, subjectivity-based adjective orderings yield a higher probability of successful reference resolution.


\textbf{Keywords:} 
adjective ordering, subjectivity, reference resolution, hierarchical modification

\end{abstract}

\section{Introduction}

When speakers use two or more adjectives to modify a noun, they exhibit robust preferences in the relative order of the adjectives (e.g., \emph{big blue box} vs.~\emph{blue big box}). These preferences surface also in listeners as they encounter multi-adjective strings. Using a series of behavioral and corpus experiments, \citeA{scontrasetal2017adjectives} demonstrated that adjective order in multi-adjective strings is reliably predicted by the subjectivity of the adjectives involved: less subjective adjectives are preferred closer to the modified noun, and the strength of the preference is modulated by the subjectivity differential between the adjectives. Thus, speakers and listeners strongly prefer \emph{big blue box} over \emph{blue big box}, as \emph{blue} is much less subjective than \emph{big}.

The question that immediately arises is why subjectivity should play the role it does in adjective ordering preferences. The current work follows \citeA{simonic2018} and \citeauthor{scontrasetalSPadjectives} (to appear) in advancing the claim that pressures from successful reference resolution deliver subjectivity-based ordering preferences. In cases of restrictive modification, adjectives that compose with the nominal later will classify a smaller set of potential referents (e.g., the set of boxes vs. the set of blue boxes). To avoid alignment errors where a listener might mis-characterize the intended referent, speakers introduce the more error-prone (i.e., more subjective) adjectives later in the hierarchical construction of nominal structure; the structure linearizes such that subjectivity decreases the closer you get to the modified noun. 
We improve on the work that precedes ours by making minimal assumptions about online processing (cf.~\citeauthor{scontrasetalSPadjectives}, to appear) and by assuming a more realistic implementation of adjective subjectivity within an empirically-motivated semantics (cf.~\citeNP{simonic2018}).



\section{Background}

\paragraph{\citeA{scontrasetal2017adjectives}}

\paragraph{\citeA{simonic2018}}

\paragraph{\citeauthor{scontrasetalSPadjectives} (to appear)}

\paragraph{\citeA{schmidtetal2009}}



\bibliographystyle{apacite}
\setlength{\bibleftmargin}{.125in}
\setlength{\bibindent}{-\bibleftmargin}

\bibliography{adjOrder}

\end{document}

